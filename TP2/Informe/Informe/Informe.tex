\input{Header.tex}

\begin{document}

%%%%%%%%%%%%%%%%%%%%%%%%%
%		Caratula		%
%%%%%%%%%%%%%%%%%%%%%%%%%

\begin{titlepage}
\newcommand{\HRule}{\rule{\linewidth}{0.5mm}}
\center
\mbox{\textsc{\LARGE \bfseries {Instituto Tecnológico de Buenos Aires}}}\\[1.5cm]
\textsc{\Large 22.12 Electrónica II}\\[0.5cm]


\HRule \\[0.6cm]
{ \Huge \bfseries Trabajo práctico N$^{\circ}$2}\\[0.4cm] 
\HRule \\[1.5cm]


{\large

\emph{Grupo 1}\\
\vspace{3px}

\begin{tabular}{lr} 	
\textsc{Mechoulam}, Alan  &  58438\\
\textsc{Lambertucci}, Guido Enrique  & 58009 \\
\textsc{Rodriguez Turco}, Martín Sebastian  & 56629 \\
\textsc{Londero Bonaparte}, Tomás Guillermo  & 58150 \\
\textsc{Scala}, Tobías & 55391 \\
\end{tabular}

\vspace{20px}

\emph{Profesores}\\
\textsc{Hirchoren}, Gustavo Abraham\\
\textsc{Petrucci}, Javier David\\
\vspace{3px}

\vspace{100px}

\begin{tabular}{ll}

Presentado: & 04/06/20\\

\end{tabular}

}

\vfill

\end{titlepage}


%%%%%%%%%%%%%%%%%%%%%
%		Indice		%
%%%%%%%%%%%%%%%%%%%%%
\vspace{3.3cm}
\tableofcontents
\newpage

%%%%%%%%%%%%%%%%%%%%%
%		Informe		%
%%%%%%%%%%%%%%%%%%%%%
\null
\vspace{8.5cm}
\section*{Consideraciones de Diseño}
	\label{consideraciones}
	
Se debe diseñar un amplificador clase AB de audiofrecuencia para una carga nominal de $8 \ \Omega$. El amplificador debe disipar la máxima potencia a la salida sin recorte para una señal de entrada de $1 V_{pp}$. Se quiere una impedancia de entrada de $50 \ k\Omega$ y corriente de reposo de los transistores de salida ajustable. Además, se consideró inicialmente una potencia máxima de $12 \ W$. Sin embargo, tras consultar acerca de la potencia máxima, fue recomendado disipar una potencia mucho mayor. Se considera finalmente $1 \ kW$ de potencia máxima a la salida sin recorte. El diseño se centra en lograr esta potencia máxima con una THD menor a $1\%$. Además, se debe diseñar para maximizar el rendimiento.
	
	
\newpage
\null
\vspace{4.75cm}
\section{Características del Amplificador}
	\label{caracteristicas}
	
El circuito final es el que se presenta a continuación:
 \begin{figure}[H]
\centering
	\includegraphics[width=\textwidth]{ImagenesCaracteristicas/TEX1.pdf}
	\caption{Etapas de entrada y amplificación}
	\label{fig:circ}
\end{figure}
 \begin{figure}[H]
\centering
	\includegraphics[width=\textwidth]{ImagenesCaracteristicas/TEX2.pdf}
	\caption{Etapas de salida y carga}
	\label{fig:circ}
\end{figure}\footnote{Vale mencionar que esta imagen al ser vectorizada se le puede hacer zoom sin que se vea borrosa.}

Ahora se mencionan algunas características notables del circuito:
\begin{itemize}
\item Cuentas con una potencia máxima de 1kW
\item Configuración de puente H  para la carga.
\item Rendimiento del amplificador balanceado del 74.5$\%$
\item En los extremos de la banda audible, correspondientes a la respuesta en frecuencia es de -3dB respecto a la banda de paso.

\item El circuito se mantiene estable frente a ruido de distinta naturaleza y mantiene una salida senoidal.

\item Cuenta con una impedancia de entrada de 50k$\Omega$ en toda la banda de paso.
\item La impedancia de salida es pequeña, del orden de las decenas de m$\Omega$, aunque el valor óptimo para esta es de 8$\Omega$ para que haya máxima transferencia de potencia.


\item Cuenta con una distorsión armónica del $\%$
\item Se obtiene una tensión de $\pm$90$\hat{V}$ con una fuente de  alimentación de $\pm 60V$
\item El circuito cuenta con la cualidad de ajustar el nivel de continua sobre la carga.
\item Permite calibrar el nivel de continua en reposo sobre los transistores de salida.
\item Implementación con una única fuente partida.



\end{itemize}



\newpage

\section{Etapa de Entrada}
	\label{inputstage}
	\subsection{Introducción}

La etapa de entrada de un amplificador cumple con la función de restar a la entrada la señal proveniente de la realimentación, para así obtener la señal deseada de error, a partir de la cual se confecciona la salida del sistema.

\subsection{Distorsión}

La distorsión puede ser provocada o provenir de distintas fuentes. Los amplificadores que se valen del uso de pares diferenciales como entrada se caracterizan por poseer un bajo offset de continua, dado a la cancelación de los voltages de $V_{BE}$. Pero mucho más notorio e importante es que la corriente de mantenimiento no atraviesa la red de realimentación. Finalmente, una ventaja también a destacar es que posee una linealidad superior a las entradas basadas en un solo transistor. Se puede observar la comparación de distorsión entre una etapa diferencial frente a una individual en la Figura (\ref{fig:thd1}).

Es muy importante reflejar que esta etapa debe la que posee la mínima distorsión, por sobre las demás, ya que las señales que maneja son pequeñas, dándose el aumento de estas en la etapa de amplificación.    
\begin{figure}[H]
\centering
	\includegraphics[width=0.6\textwidth]{ImagenesInput-Stage/thd1.PNG}
	\caption{Comparación de THD en función de la frecuencia.}
	\label{fig:thd1}
\end{figure}

Si bien, seleccionando adecuadamente las resistencias del circuito se puede balancear el par, quedan pendientes ciertos parámetros. Las corrientes de colectores deben ser lo más similares posibles. Debe existir una precisión del $1\%$ o mejor para poder optimizar la linealidad del la etapa, y de esta forma, reducir la distorsión en altas frecuencias.
\begin{figure}[H]
\centering
	\includegraphics[width=0.6\textwidth]{ImagenesInput-Stage/thd2.PNG}
	\caption{THD en función de la entrada (en $dBu$, nivel de referencia $0.7746 \ V$) al variar el balance de la corriente del colector . Número de curva y distorsión especificadas en la Tabla (\ref{tab:thd2}).}
	\label{fig:thd2}
\end{figure}

\begin{table}[H]
\centering
\begin{tabular}{cc}
\hline
\textbf{Número de curva} & \textbf{Variación $\mathbf{I_C}$ [\%]} \\ \hline
1                        & 0                    \\
2                        & 0.5                  \\
3                        & 2.2                  \\
4                        & 3.6                  \\
5                        & 5.4                  \\
6                        & 6.9                  \\
7                        & 8.5                  \\
8                        & 10             		\\
\hline
\end{tabular}
\caption{Número de curva y distorsión de la Figura (\ref{fig:thd2}).}
\label{tab:thd2}
\end{table}

Una opción valida (e implementada en este diseño) se basa en el uso de una fuente de corriente compensada para polarizar el par. Además, se emplean resistencias en los emisores para ajustar dicho parámetro. Es importante recordar que dichos elementos no deben ser lo suficientemente grandes como para que afecte el ruido térmico\footnote{D. Self, Audio Power Amplifier Design Handbook, 5ta ed. Kidlington: Elsevier Science, 2014.}.  

\subsection{Linealidad y corriente de colector}

La transconductancia aumenta con la corriente de colector. Elevar este último parámetro es posible y relativamente sencillo. Una técnica empleada para poder mejorar la linealidad en altas frecuencias consiste en aumentar la corriente mencionada para luego reducirla a través del lazo de realimentación negativa. Esta no linealidad es atribuida a la resistencia del $R_E$ del emisor, la cual no es una resistencia física, sino que una resultante de la expresión de la pendiente de la corriente de colector. 

La corriente de mantenimiento a la entrada es uno de los parámetros que define el máximo slew rate (SR). Otro factor importante que lo limita es el polo dominante proveniente del capacitor de Miller. Este último es solucionado por los requerimientos que se deben cumplir para conseguir la estabilidad. Por otro lado, aumentar la corriente de colector puede aumentar el factor de SR sin afectar la estabilidad, siempre y cuando, la transconductancia se mantenga en el valor deseado. 

A pesar de ello, existen límites para esta corriente. El aumento de las corrientes de bias, como la caída de tensión a través de las resistencias son algunos ejemplos. El factor más limitante es la potencia disipada a lo largo de esta etapa, ya que no siempre deja margen para incrementar la $I_C$. 

\subsection{Ruido}
El ruido existente en la etapa de entrada surge de los componentes activos y las resistencias que se presentan en la entrada. Las condiciones de operación de los transistores se encuentran limitadas por los factores de SR y linealidad. Por otro lado, ya que el ruido es una función que dependen de $I_C$, bastan con ajustar dicho parámetro para poder reducir el ruido. 

Además de reducir las resistencias existentes en la etapa de entrada para así reducir el ruido térmico existente, es posible realizar el mismo efecto haciendo lo mismo con la resistencia observada a la salida del lazo de realimentación negativa. Este último paso es más delicado ya que puede generar grandes cambios en el sistema.

\subsection{Limitación en banda}
Se sabe que  existen etapas que acumulan cambios de la fase, siendo estas las altas frecuencias, las cuales tienden a ser más inestables y generar oscilaciones. Es así que se puede llegar a dañar los dispositivos a la salida del sistema por sobrecalentamiento, entre otros motivos. Esto es causado por la distorsión del amplificador y el incremento de la ganancia de lazo abierto. Limitar en banda esta ganancia evita que la señal del realimentador.

\subsection{Calculo de componentes}
Para poder seleccionar los elementos y parámetros que componen la etapa de entrada, se comenzó recorriendo la malla del par. Fijando las caídas de tensión $V_{CE} = 10 \ V$, se obtiene:
\begin{equation}
	15 \ V - \frac{I_O}{2} R_C - 10 \ V - 10 \ V - 0.7 \ V + 15 \ V = 0
\end{equation}

Ya que se polariza ambos transistores por una fuente de corriente compensada, se obtiene para esa configuración la siguiente ecuación:
\begin{equation}
	-15 \ V - 2V_{D} + V_{BE} + 15 \ V = I_O R 
\end{equation}

Además, se coloca un preset en los emisores del par para poder ajustar la corriente que circula, en caso de ser necesario. Dado esto último, la ganancia del par diferencial con el preset viene dada por la expresión:
\begin{equation}
	\Delta V_D = \frac{h_{fe} R_D}{2 h_{ie} + h_{fe} R_V} = \frac{R_D}{2 \frac{V_T}{I_C} + R_V}
\end{equation}

Sabiendo que la resistencia del colector se encuentra en paralelo con una resistencia de $390 \ \Omega$, y mediante el uso de las ecuaciones previamente mencionadas, se obtienen los siguientes valores de interés:

\begin{table}[H]
\centering
\begin{tabular}{cccc}
\hline
$\mathbf{I_O}$ & $\mathbf{R}$  & $\mathbf{R_C}$ & $\mathbf{R_E}$ \\	\hline
$-9.12 \ mA$   & $68 \ \Omega$ & $2 \ k\Omega$  & $1 \ k\Omega$ \\
\hline
\end{tabular}
\caption{Parámetros de la etapa de entrada.}
\end{table}
		
\section{Etapa de Ganancia}
	\label{gainstage}
	%\input{../Informe/Header.tex}
%
%\begin{document}

\subsection{Introducción}
La etapa de amplificación de tensión (o VAS) a menudo es considerada como la etapa más crítica de un amplificador de potencia , dado que no solo provee la ganancia de tensión sino que también debe manejar todo el rango de la tensión de salida. Esto indicaría que puede jugar un rol significativo en la distorsión armónica de la señal, sin embargo un VAS bien diseñado contribuye relativamente poco a la distorsión total, e incluso si se tomasen pasos extra para intentar linealizar aún más la salida, estas contribuciones comparadas con las hechas en una etapa de entrada, son completamente despreciables.\\
\subsection{Diseños propuestos}
El primer diseño que se pensó fue un emisor común que se observa a continuación que si bien  cuenta con varios problemas estos van  a ser sorteados en las siguientes lineas.
\begin{figure}[H]
\centering
	\includegraphics[width=0.5\textwidth]{ImagenesGain-Stage/ec1.png}
	\caption{Primer diseño Emisor Común}
	\label{fig:ec1}
\end{figure}

El primer problema que se afrontó fue el de limitar la ganancia de altas frecuencias.
Para esto se introduce una linea de realimentación negativa utilizando $C_{dom}$ el cual limita la ganancia de altas frecuencia y así asegura mayor estabilidad.
 \begin{figure}[H]
\centering
	\includegraphics[width=0.5\textwidth]{ImagenesGain-Stage/ec2.png}
	\caption{Segundo diseño Emisor Común}
	\label{fig:ec2}
\end{figure}
En cuanto al cálculo de $C_{in}$ o $C_{out}$ se buscó que la impedancia para frecuencias medias sea despreciable frente a la impedancia vista desde el colector o la base.

Para el caso de $C_{Dom}$ se eligió un valor tal que las altas frecuencias vean un camino de baja impedancia comparado con la entrada del emisor común. De esta manera, las frecuencias altas no serán amplificadas.

Es importante que la ganancia a lazo abierto del VAS sea alta, así este puede ser linealizado. Si se intenta aumentar la ganancia del emisor común, se sabe que la ganancia de este está descripto por la ecuación:
\begin{align}
A_{vs}=\frac{-R_D}{R_E}
\end{align}

Si se quiere subir la ganancia, puede subirse el valor de $R_C$, pero esto dado una $I_c$ determinada por la  malla de entrada, provocará que caiga más tensión sobre la resistencia $R_C$ y así consuma un valor de potencia mucho más elevado. Una manera de asegurar una gran ganancia es utilizar una carga activa con una fuente de corriente, así suministrando la corriente necesaria, y mostrando una alta impedancia.

Además, se optó por utilizar un acople por fuente de corriente en vez de uno capacitivo con la idea de no introducir singularidades no deseadas en el circuito.
 \begin{figure}[H]
\centering
	\includegraphics[width=0.5\textwidth]{ImagenesGain-Stage/ec3.png}
	\caption{Tercer diseño Emisor Común}
	\label{fig:ec3}
\end{figure}

Para la fuente de corriente de la carga activa se optó por utilizar una fuente compensada. Mientras que para el acople se utilizó una fuente implementada con JFET, debido a que en esa zona el circuito tiene la máxima variación de tensión, lo cual si se hubiese implementado con una fuente con BJT podría tener problemas de Saturación / Corte.\\
El transistor se le impuso una tensión de $V_{ce}$ de $\frac{V_{cc}+V_{ee}}{2}$ al igual que un valor de corriente $I_c = 10 \ mA$, al igual que optar por una resistencia de emisor baja.\\
Los valores utilizados para el circuito responden a las siguientes ecuaciones:
\begin{align}
R_{B2}= \frac{V{Re}+V_{be}}{I_{RB}} \approx 390 \ \Omega |_{IRb=5 \ mA \  \ \wedge \ \  Vre=1.2 \ V}
\end{align}
\begin{align}
R_E=\frac{I_{Rb}\cdot R_{B2}-2V_{be}}{I_c}=120 \ \Omega
\end{align}
\begin{align}
R_{B1}= \frac{V_{cc}+V_{ee}-I_{Rb}\cdot R_{B2} }{I_{Rb}}=24 \ k\Omega
\end{align}
Luego para la fuente compensada se obtuvo un valor para la resistencia $R_{cc}$, tal que suministre los $10 \ mA$ al colector del transistor.

\begin{align}
R_{cc}=\frac{V_{be}}{I_c}\approx 76 \ \Omega
\end{align}
También se obtiene el parámetro de la impedancia de salida de la fuente compensada la cual es de aproximadamente:
\begin{align}
R_{of} = (R_cc // hie^*)+(1+hfe^* )\cdot rce
\end{align}

En cuanto a la compensación por corriente se optó por el transistor \href{https://ar.mouser.com/datasheet/2/827/DS_UJ3N065080K3S-1530401.pdf}{UJ3N065080K3S} debido a que puede manejar el rango de tensiones de la salida y puede proveer una corriente suficiente, si bien este era el transistor ideal, el subcircuito encontrado en linea parece no funcionar correctamente por lo que se utilizo el \href{http://www.linearsystems.com/lsdata/datasheets/LSK489_LOW_NOISE,_LOW_CAPACITANCE_MONOLITHIC_DUAL_N-CHANNEL_JFET.pdf}{LSK489B} para la simulación, pero ese transistor sería el utilizado en la realidad. En cuanto a la  elección de la resistencia del Jfet, se consideraron los parámetros $V_p$, $I_{dss}$.
\begin{align}
I_D=-\frac{V_{GS}}{R_j}
\end{align}
El valor de la corriente de drain que se necesitará será la corriente la cual se fugaría del VAS hacia la etapa de salida al no estar presente el capacitor, la cual es de un valor de aproximadamente $5 \ mA$.
\begin{align}
I_D= I_{DSS} \cdot \left(1-\frac{V_{GS}}{V_P} \right)^2
\end{align}

De aquí se obtiene $R_j \approx 65 \ \Omega$. La fuente de corriente de acople posee además una segunda característica. Esta permite no solo regular el nivel de continua de reposo a la salida de cada clase AB, sino también el nivel de continua de reposo sobre la carga. Basta que ambas salidas en reposo sean iguales para que la continua de reposo sobre la carga sea nula, sin embargo, si no se compensa correctamente con esta fuente el nivel de continua en reposo a la salida de cada clase AB, se consumirá una gran potencia lo que disminuirá considerablemente el rendimiento del amplificador, por más que los parlantes no se dañen debido a la simetría del circuito.

Como última consideración se optó por poner 3 etapas amplificadoras como la de la Figura (\ref{fig:ec1}) previo al diseño de la Figura (\ref{fig:ec3}) con la intención de aumentar la ganancia del bloque A tal  que valga que $\alpha \cdot \beta \gg 1$ dado que $\beta$ será de un valor muy pequeño para lograr llegar a disipar como máximo $1.5 \ kW$ sobre la carga.
Con esto dicho la tension sobre los $V_{ce}$ serán:\\
Para el transisitor con carga activa:
\begin{align}
V_{ce-ec-rms}= V_{ce-cc}+\frac{ \hat{V_o}}{\sqrt{2}}\approx 68.9V 
\end{align}
Para la carga activa será:
\begin{align}
V_{ce-load-rms}= V_{cc}-I_C-\frac{\hat{V_{ce-ec}}}{\sqrt{2}}\approx 71.5V
\end{align}


%\begin{align}
%\end{align}
%\end{document}

	
\section{Etapa de Salida}
	\label{outputstage}
	\input{../Informe/Header.tex}

\begin{document}

\subsection{Introducción}

La etapa de salida de un amplificador de audio se encarga de entregar la corriente necesaria a la carga para conseguir la característica de potencia buscada en el amplificador; sin así distorsionar demasiado a la señal, para preservar el THD para la cual se trabajo tanto para disminuir en las etapas anteriores.

Por un lado, se puede utilizar tecnología FET, los cuales no poseen una reducción del "beta" frente a grandes corrientes y por ende una mayor linealidad, a un costo de duplicar el precio de cada dispositivo activo. La otra alternativa, mucho más popular, es usar tecnología BJT. Luego existen varias clases de etapa de salidas distintas, entre ellas A, con una alta linealidad pero muy baja eficiencia; clase B, la cual soluciona el problema de la eficiencia, al costo de la distorsión por crossover; la clase AB, un compromiso entre ambas; variaciones de la popular clase AB, como la clase G, y muchas más.
En el amplificador de audio diseñado se utilizará clase AB al ser una consideración de diseño.

Dentro de la clase AB existen muchas topologías distintas cada una con sus respectivas ventajas y desventajas. Una de ellas es la topología EF (emitter-follower) compuesta por dos emisores comunes en cascada. En esta configuración, un transistor funciona como \textit{driver}, generalmente situado en un punto Q muy estable. Este transistor proporciona la corriente de base al siguiente, por el cual fluye la corriente que se le entrega a la carga. Dentro de esta topología existen tres tipos mostrados en la Figura (\ref{fig:ef}). En el tipo 1 los resistores de emisor se conectan a la salida. Estos resistores colocan al driver en un punto Q estable. El tipo 2 posee la ventaja de ahorrarse un resistor, colocando uno solo entre los emisores de los transistores. Además, los transistores nunca se polarizarán en inversa al pasar de un semiciclo al otro. En el tipo 3, se conectan los resistores a los rieles de alimentación, lo cual puede mejorar el apagado de altas frecuencias.

Otra configuración involucra los pares Sziklai, Quasi-Darlington o también llamado par de realimentación, dado que ahora el driver se coloca de manera tal que este compare la tensión a la salida con la entrada, lo cual aumenta la linealidad. Además, como el Vbe del transistor de salida se encuentra dentro del lazo de realimentación, se observa una estabilidad térmica mayor.

Una topología la cual en el pasado era casi obligatoria por la falta de transistores PNP de potencia complementarios a los NPN. En esta configuración solamente se reemplaza por un par Quasi-Darlington a los transistores PNP. Esta topología presenta una alinealidad mucho mayor y no será utilizada, aunque existen varios arreglos a la simetría, como por ejemplo utilizando un diodo de Baxandall.

Naturalmente surge al analizar estas etapas de salida la idea de colocar tres transistores en cascada, o más. De ahí surge la topología Triple EF. Esta configuración presenta mayor linealidad a alta potencias, un punto Q más estables para los transistores \textit{pre-drivers}, los que proporcionan la corriente a los drivers, debido a que estos manejarán una corriente menor y disiparán menor potencia. Además, al poseer una ganancia de corriente mayor, la etapa de ganancia deberá proporcionar corrientes menores. En la Figura (\ref{triples}) se observan distintas configuraciones Triple EF posibles.

189

\subsubsection{Topología Utilizada}


\subsubsection{Generador de Bias}

\subsubsection{Fuente de corriente}

\end{document}
	
\section{Alimentación}
	\label{alimentacion}
	\input{../Informe/Header.tex}

\begin{document}
\subsection{Alimentación}
Se eligió utilizar una fuente partida de $\pm$60V  debido a que se buscaba una tensión de salida elevada para obtener un valor de potencia sobre la carga de $\approx 1KW$, para esto la tensión en modo diferencial debe ser de $\approx$ 90V, defininedo asi la tensión de alimentación.\\
También se optó por utilizar un segundo riel de alimentación para el par diferencial, teniendo como objetivo optimizar el rendimiento, dado que este trabaja con pequeña señal. Se optó por un valor de $\pm$ 15V, utilizando una resisitencia y un diodo zener para proveer esa tensión a partir de la fuente partida principal.
\begin{figure}[H]
\centering
	\includegraphics[width=0.7\textwidth]{ImagenesAlimentacion/al.png}
	\caption{Fuente de alimentación}
	\label{fig:alimentacion}
\end{figure}
El diodo seleccionado es uno cuya tensión de zener es de 15V.
Para el cálculo de la resistencia del diodo se tuvo en cuenta que el diodo quede polarizado con una corriente de mantenimiento de 6mA al igual que haya suficiente corriente en la rama para que el par diferencial utilice. Teniendo en cuenta la corriente de polarización del par diferencial se llega  ala conclusión de que la corriente por la resistencia debe ser de 20mA.
\begin{align}
R1=\frac{60-V_z}{20mA}= 2k2\Omega
\end{align}
\end{document}

\section{Cálculos de Potencia}
	\label{calculospotencia}
	\input{../Informe/Header.tex}

\begin{document}

\subsection{Introducción}

\subsection{Simulación de  rendimiento}
El rendimiento esta definido como:
\begin{align}
\eta = \frac{P_{RL}}{P_{vcc}+P_{vee}}
\end{align}
Teniendo en cuenta que la potencia para las señales senoidales se toma la potencia eficaz, siendo esta:
\begin{align}
P_{R}=V_{R-RMS}\cdot I_{R-RMS} =\frac{\hat{V_{R}}}{\sqrt{2}} \cdot \frac{\hat{I_{R}}}{\sqrt{2}}
\end{align}
Finalmente fue simulado el rendimiento obteniendo los siguientes valores:
\begin{align}
\eta=\frac{1083 / 2 W}{ 2\cdot 18 W + 2 \cdot 690 /2}\approx 74.5 \%
\end{align}
\end{document}	
	
\section{Simulaciones}
	\label{simulaciones}
	\input{../Informe/Header.tex}


\begin{document}

\subsection{Introducción}
Se realizaron simulaciones en LTSpice del circuito propuesto, así también se comprobó que los resultados teóricos concuerdan con las simulaciones, ademas se tuvo especial cuidado a la hora de evaluar que transistores y resistores usar en las etapas tal que no haya problemas de potencia, aqui se muestran las tensiones y potenicas relevantes del circuito en cuanto a la elección crítica de componentes.\\
Comenzando por los emisores comunes la potencia se encuentra cerca del máximo y la tensión Vce en un rango seguro.

\begin{figure}[H]
	\centering
	\includegraphics[width=0.8\textwidth]{ImagenesSimulaciones/PEC1.png}
		\includegraphics[width=0.8\textwidth]{ImagenesSimulaciones/VEC1.png}
	\caption{Potencia y tensión sobre un emisor común sin carga activa.}
	\label{fig:pec1}
\end{figure}También se cuenta con que la potencia disipada por las resistencias de colector son cercanas al medio watt, por lo que las resistencias a utilizar son de medio watt.\\
Luego el úlitmo emisor común, el cual tiene la mayor ganancia de todos, es razonable esperar que disipe mas potencia y en efecto así es, por eso la necesidad de utilizar un transistor de mayor potencia para este ademas de que este manejara la máxima variación de tensión.

\begin{figure}[H]
	\centering
	\includegraphics[width=0.8\textwidth]{ImagenesSimulaciones/PECF.png}
		\includegraphics[width=0.8\textwidth]{ImagenesSimulaciones/VECF.png}
	\caption{Potencia y tensión sobre un emisor común con carga activa.}
	\label{fig:pecf}
\end{figure}
%% potencai fuente de corriente de la carga activa
En cuanto a la carga activa, dado que se encuentra en el colector del transistor va a  sufrir también una gran variación de tensión por lo que se necesita un transistor que pueda manejar dicha tensión.
\begin{figure}[H]
	\centering
	\includegraphics[width=0.8\textwidth]{ImagenesSimulaciones/PCSEC.png}
		\includegraphics[width=0.8\textwidth]{ImagenesSimulaciones/VCSEC.png}
	\caption{Potencia y tensión sobre la carga activa.}
	\label{fig:pcsecf}
\end{figure}

%% potencia de la fuente de corriente del multiplicador de vbe
Continuando por la potencia y tensión correspondiente a la fuente de corriente de la etapa de salida.
\begin{figure}[H]
	\centering
	\includegraphics[width=0.8\textwidth]{ImagenesSimulaciones/PCSVBE.png}
		\includegraphics[width=0.8\textwidth]{ImagenesSimulaciones/VCSVBE.png}
	\caption{Potencia y tensión de la fuente de corriente de la salida.}
	\label{fig:pcsvbe}
\end{figure}
%%potencia en el primer transistor de salida
Considerando la potencia de la etapa de salida consideraremos la potencia de las 3 fases, correspondiendo a la primera, la de menor potencia:
\begin{figure}[H]
	\centering
	\includegraphics[width=0.8\textwidth]{ImagenesSimulaciones/PO1.png}
		\includegraphics[width=0.8\textwidth]{ImagenesSimulaciones/VO1.png}
	\caption{Potencia y tensión del primer transistor de salida.}
	\label{fig:po1}
\end{figure}
Para el segundo, el cual corresponde a mediana potencia:
\begin{figure}[H]
	\centering
	\includegraphics[width=0.8\textwidth]{ImagenesSimulaciones/PO2.png}
		\includegraphics[width=0.8\textwidth]{ImagenesSimulaciones/VO2.png}
	\caption{Potencia y tensión del segundo transistor de salida.}
	\label{fig:po2}
\end{figure}
Finalmente para los últimos transistores, los cuales son los que trabajan con la mayor parte de la potencia de salida se obtuvo tanto la potencia como la tensión sobre ellos.
\begin{figure}[H]
	\centering
	\includegraphics[width=0.8\textwidth]{ImagenesSimulaciones/PO3.png}
		\includegraphics[width=0.8\textwidth]{ImagenesSimulaciones/VO3.png}
	\caption{Potencia y tensión de los transistores de salida.}
	\label{fig:po3}
\end{figure}
Adicional mente se puede observar que en el gráfico de potencias se dibujo también la potencia del transistor PNP de la rama inferior, para hacer evidente el nivel de simetría que se tiene.\\
Finalmente mostraremos la simulación de potencia sobre la carga.
\begin{figure}[H]
	\centering
	\includegraphics[width=0.8\textwidth]{ImagenesSimulaciones/PRL.png}
	\caption{Potencia sobre la carga.}
	\label{fig:porl}
\end{figure}
Aquí se pueden apreciar varias cosas, primeramente que con una tensión de entrada de 0.5$\hat{V}$ se obtiene la mayor potencia, la cual corresponde a $\approx 1078W$, también debe notarse que la potencia entregada por la fuente es $\approx 0W$ cuando la potencia sobre la carga también lo es.\\
Luego se obtuvo la salida del circuito teniendo a la entrada una señal senoidal de frecuencia 1KHz y de amplitud 0.25$\hat{V}$ y otra simulación con la misma entrada pero sumado también ruido blanco con distribución uniforme de amplitud $25m\hat{V}$.
\begin{figure}[H]
	\centering
	\includegraphics[width=0.8\textwidth]{ImagenesSimulaciones/VRL.png}
		\includegraphics[width=0.8\textwidth]{ImagenesSimulaciones/VRLNoise.png}
	\caption{Tensión sobre la carga con y sin ruido.}
	\label{fig:VRLN}
\end{figure}
Finalmente se obtuvo el Bode del sistema como se observa a continuación:
\begin{figure}[H]
	\centering
	\includegraphics[width=0.9\textwidth]{ImagenesSimulaciones/BODE.png}
	\caption{Bode del sistema.}
	\label{fig:bode}
\end{figure}
Donde se puede apreciar una caida de 3dB respecto a la banda pasante tanto en 20Hz como en 20KHz
%%Bode
\end{document}
	
\section{Cálculos de Disipador}
	Dado que los transistores driver y de salida consumen una potencia de $1.41 \ W$ y $13 \ W$ en promedio, se debió realizar el cálculo de disipadores para estos. Para los transistores de salida se decidió utilizar un disipador de extrusión para poner a todos estos en un mismo disipador. Para un disipador con $N$ transistores, se puede considerar, si se deja un margen de espacio entre cada transistor, que estos ven la resistencia térmica del disipador $N$ veces más grande. Luego, se tiene que
	
	\begin{equation}
	\frac{T_j - T_a}{R_{\theta jc} + N \cdot R_{\theta sa}} = 13 \ W
	\end{equation}
	
Considerando una $T_j$ máxima de $170  C$, $30 C$ menos que la máxima según la datasheet de los transistores; una temperatura de ambiente de $45 C$; la $R_{\theta jc}$ de $0.7 \ \frac{ C}{W}$ proporcionada por el fabricante; que se dejará un espacio entre transistores equivalente a un mismo transistor de encapsulado TO-3 de ancho de 1 pulgada; y que se pondrán 12 transistores en un mismo disipador
	
	\begin{equation}
	\frac{170  C - 45  C}{0.7 \frac{ C}{W} + 12 \cdot R_{\theta sa}} = 13 \ W
	\end{equation}	
	
	\begin{equation}
	R_{\theta sa} = 0.743 \ \frac{ C}{W}
	\end{equation}
	
Por lo que se utilizarán dos disipadores ATS-EXL109 de 24 pulgadas de largo de extrusión junto al ventilador F16EA-03LLC/E de $3 \ V$ de continua para disminuir la resistencia térmica de los disipadores a $0.71 \ \frac{ C}{W}$ lo que deja un buen margen de seguridad si se considera que por lo general la música posee una potencia media menor a una pura senoidal constante, con lo cual se calcularon estos valores.
Finalmente, para los transistores driver, se tiene que

	\begin{equation}
	\frac{T_j - T_a}{R_{\theta jc} + N \cdot R_{\theta sa}} = \frac{170  C - 45  C}{ 0.7 \frac{ C}{W} + \cdot R_{\theta sa}} = 1.41 \ W
	\end{equation}
	
		\begin{equation}
	R_{\theta sa} = 87.95 \ \frac{ C}{W}
	\end{equation}

Por lo que no necesitará disipador.	
	
\section{Conclusiones}
	\label{simulaciones}
	\input{../Informe/Header.tex}

\begin{document}	

Se consiguió elaborar un amplificador para una carga nominal de $8 \ \Omega$, con una disipación máxima de $1.5 \ kW$ y una distorsión armónica no mayor a $0.529\%$. Es por ello que cabe preguntarse, ¿es posible mejorar el circuito?. Desde nuestro punto de vista, se llegó a la conclusión de que sí se puede mejorar ciertos parámetros, como por ejemplo, la distorsión armónica. La mejora propuesta consiste en el uso de amplificadores operacionales dentro del circuito. Estos permiten reducir el TDH a $0.015 \%$. Otra ventaja que presenta esta implementación es que se puede aplicar una realimentación negativa diferencial mediante el uso de la configuración sumador-restador. Este tipo de realimentación permite subir ..
	
\end{document}
	
\end{document}