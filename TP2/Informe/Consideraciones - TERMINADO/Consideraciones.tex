\documentclass[a4paper]{article}
\usepackage[utf8]{inputenc}
\usepackage[spanish, es-tabla, es-noshorthands]{babel}
\usepackage[table,xcdraw]{xcolor}
\usepackage[a4paper, footnotesep = 1cm, width=20cm, top=2.5cm, height=25cm, textwidth=18cm, textheight=25cm]{geometry}
%\geometry{showframe}

\usepackage{tikz}
\usepackage{amsmath}
\usepackage{amsfonts}
\usepackage{amssymb}
\usepackage{float}
\usepackage{graphicx}
\usepackage{caption}
\usepackage{subcaption}
\usepackage{multicol}
\usepackage{multirow}
\setlength{\doublerulesep}{\arrayrulewidth}
\usepackage{booktabs}

\usepackage{hyperref}
\hypersetup{
    colorlinks=true,
    linkcolor=blue,
    filecolor=magenta,      
    urlcolor=blue,
    citecolor=blue,    
}

\newcommand{\quotes}[1]{``#1''}
\usepackage{array}
\newcolumntype{C}[1]{>{\centering\let\newline\\\arraybackslash\hspace{0pt}}m{#1}}
\usepackage[american]{circuitikz}
\usetikzlibrary{calc}
\usepackage{fancyhdr}
\usepackage{units} 

\graphicspath{{../Calculos-Potencia/}{../Caracteristicas/}{../Consideraciones/}{../Gain-Stage/}{../Input-Stage/}{../Output-Stage/}{../Simulaciones/}{../Alimentacion/}{../Conclusiones/}}

\pagestyle{fancy}
\fancyhf{}
\lhead{22.12 Electrónica II}
\rhead{Mechoulam, Lambertucci, Rodriguez, Londero, Scala}
\rfoot{Página \thepage}

\begin{document}

Se debe diseñar un amplificador clase AB de audiofrecuencia para una carga nominal de $8 \ \Omega$. El amplificador debe disipar la máxima potencia a la salida sin recorte para una señal de entrada de $1 V_{pp}$. Se quiere una impedancia de entrada de $50 \ k\Omega$ y corriente de reposo de los transistores de salida ajustable. Además, se consideró inicialmente una potencia máxima de $12 \ W$. Sin embargo, tras consultar acerca de la potencia máxima, fue recomendado disipar una potencia mucho mayor. Se considera finalmente $1 \ kW$ de potencia máxima a la salida sin recorte. El diseño se centra en lograr esta potencia máxima con una THD menor a $0.05\%$. Además, se debe diseñar para maximizar el rendimiento.
 
\end{document}