%\input{../Informe/Header.tex}
%%
%\begin{document}

%\subsection{Introducción}

\subsection{Simulación de  rendimiento}
El rendimiento esta definido como:
\begin{align}
\eta = \frac{P_{RL}}{P_{vcc}+P_{vee}}
\end{align}
Teniendo en cuenta que la potencia para las señales senoidales se toma la potencia eficaz, siendo esta:
\begin{align}
P_{R}=V_{R-RMS}\cdot I_{R-RMS} =\frac{\hat{V_{R}}}{\sqrt{2}} \cdot \frac{\hat{I_{R}}}{\sqrt{2}}
\end{align}
Finalmente fue simulado el rendimiento obteniendo los siguientes valores:
\begin{align}
\eta=\frac{1083 / 2 W}{ 2\cdot 18 W + 2 \cdot 690 /2}\approx 74.5 \%
\end{align}

\subsection{Etapa de ganancia}
Para el cálculo de la potencia de este bastará con realizar el cálculo\footnote{Los valores de corriente $I_c$ y tensión $V_{ce}$ son los calculados en la etapa de ganancia}:
\begin{align}
P_{ec4}\approx\left[V_{ce-cc} + \frac{\hat{V_{ce-ac}}}{\sqrt{2}}\right]\cdot I_c\approx 540 \ mW
\end{align}
Dado a la potencia media que disipa y la tensión que debe manejar se eligió un TIP41C, el cual puede manejar este trabajo sin disipador.
En cuanto a las otras tensiones $V_{ce}$ de los transistores debido a que no trabajan con señales tan grande se aproxima la potencia disipada a la de continua.
\begin{align}
P_{ec-123}\approx V_{ce-cc} \cdot I_c \approx 597 \ mW
\end{align}
Este transistor todavía se encuentra dentro de la potencia que puede disipar sin disipador.
Por otro lado, para las resistencias de colector de los emisores comunes será una potencia de:
\begin{align}
P_{Rc}= I_c^2 \cdot R_c \approx 0.55 \ W
\end{align}

Para la carga a activa bastará con realizar el producto de la tension de juntura por la corriente.
\begin{align}
P_{CS}\approx\left[V_{ce-cc} + \frac{\hat{V_{ce-ac}}}{\sqrt{2}}\right]\cdot I_c\approx  632\ mW
\end{align}
En cuanto a este transistor cabe destacar que tampoco es necesario el uso de un disipador.
\subsection{Etapa de entrada}
Se calcula la potencia disipada por la etapa de entrada. Para los transistores del par diferencial, sabiendo $I_C$ y $V_{CE}$ (y despreciando la corriente de base) se obtiene aproximadamente: 
\begin{equation} 
	P_{Q65} = P_{Q66} \approx V_{CE} I_C \approx 30 \ mW
\end{equation}
para cada uno, mientras que para el transistor de la fuente de corriente
\begin{equation} 
	P_{Q9} \approx V_{CE} I_C \approx 106.23 \ mW
\end{equation}

Para el conjunto de las resistencias $R_V$ ($R_{107} + R_108$), $R_{105}$ y $R_{106}$, se obtiene una potencia de
\begin{equation} 
	P_{R105} + P_{R106} + P_{RV} \approx  I_C^2 \left(R_{105} + R_{106} + R_V \right) \approx 111.57 \ mW
\end{equation}

Para la resistencia $R_{30}$, se calculó 
\begin{equation} 
	P_{R30} \approx  I_O^2 R_{30} \approx 6.02 \ mW
\end{equation}
mientras que para $R_{31}$
\begin{equation} 
	P_{R31} \approx  \frac{\left( 30 \ V - 1.7 \ V \right)^2}{R_{31}} \approx 143.50 \ mW
\end{equation}

Luego, para los diodos $D_{19}$ y $D_{20}$, conjuntamente se calcula una potencia de
\begin{equation} 
	P_{D1} + P_{D2} \approx  \left( 1.7 \ V \right) \frac{30 \ V - 1.7 \ V}{R_{31}} \approx 7.02 \ mW
\end{equation}
 
Finalmente, sumando todas las potencias disipadas, se obtiene total de $404.34 \ mW$.  
%Este valor se contrasta contra los $433.08 \ mW$ obtenidos de la simulación. Es entendible la diferencias

%\end{document}
