\input{../Informe/Header.tex}

\begin{document}

\subsection{Introducción}

\subsection{Simulación de  rendimiento}
El rendimiento esta definido como:
\begin{align}
\eta = \frac{P_{RL}}{P_{vcc}+P_{vee}}
\end{align}
Teniendo en cuenta que la potencia para las señales senoidales se toma la potencia eficaz, siendo esta:
\begin{align}
P_{R}=V_{R-RMS}\cdot I_{R-RMS} =\frac{\hat{V_{R}}}{\sqrt{2}} \cdot \frac{\hat{I_{R}}}{\sqrt{2}}
\end{align}
Finalmente fue simulado el rendimiento obteniendo los siguientes valores:
\begin{align}
\eta=\frac{1083 / 2 W}{ 2\cdot 18 W + 2 \cdot 690 /2}\approx 74.5 \%
\end{align}
\end{document}