%\documentclass[a4paper]{article}
\usepackage[utf8]{inputenc}
\usepackage[spanish, es-tabla, es-noshorthands]{babel}
\usepackage[table,xcdraw]{xcolor}
\usepackage[a4paper, footnotesep = 1cm, width=20cm, top=2.5cm, height=25cm, textwidth=18cm, textheight=25cm]{geometry}
%\geometry{showframe}

\usepackage{tikz}
\usepackage{amsmath}
\usepackage{amsfonts}
\usepackage{amssymb}
\usepackage{float}
\usepackage{graphicx}
\usepackage{caption}
\usepackage{subcaption}
\usepackage{multicol}
\usepackage{multirow}
\setlength{\doublerulesep}{\arrayrulewidth}
\usepackage{booktabs}

\usepackage{hyperref}
\hypersetup{
    colorlinks=true,
    linkcolor=blue,
    filecolor=magenta,      
    urlcolor=blue,
    citecolor=blue,    
}

\newcommand{\quotes}[1]{``#1''}
\usepackage{array}
\newcolumntype{C}[1]{>{\centering\let\newline\\\arraybackslash\hspace{0pt}}m{#1}}
\usepackage[american]{circuitikz}
\usetikzlibrary{calc}
\usepackage{fancyhdr}
\usepackage{units} 

\graphicspath{{../Calculos-Potencia/}{../Caracteristicas/}{../Consideraciones/}{../Gain-Stage/}{../Input-Stage/}{../Output-Stage/}{../Simulaciones/}{../Alimentacion/}{../Conclusiones/}}

\pagestyle{fancy}
\fancyhf{}
\lhead{22.12 Electrónica II}
\rhead{Mechoulam, Lambertucci, Rodriguez, Londero, Scala}
\rfoot{Página \thepage}
%
%\begin{document}

%\subsection{Introducción}

\subsection{Etapa de entrada}
Se calcula la potencia disipada por la etapa de entrada. Para los transistores del par diferencial, sabiendo $I_C$ y $V_{CE}$ (y despreciando la corriente de base) se obtiene aproximadamente: 
\begin{equation} 
	P_{Q65} = P_{Q66} \approx V_{CE} I_C \approx 30 \ mW
\end{equation}
para cada uno, mientras que para el transistor de la fuente de corriente
\begin{equation} 
	P_{Q9} \approx V_{CE} I_C \approx 106.23 \ mW
\end{equation}

Para el conjunto de las resistencias $R_V$ ($R_{107} + R_108$), $R_{105}$ y $R_{106}$, se obtiene una potencia de
\begin{equation} 
	P_{R105} + P_{R106} + P_{RV} \approx  I_C^2 \left(R_{105} + R_{106} + R_V \right) \approx 111.57 \ mW
\end{equation}

Para la resistencia $R_{30}$, se calculó 
\begin{equation} 
	P_{R30} \approx  I_O^2 R_{30} \approx 6.02 \ mW
\end{equation}
mientras que para $R_{31}$
\begin{equation} 
	P_{R31} \approx  \frac{\left( 30 \ V - 1.7 \ V \right)^2}{R_{31}} \approx 143.50 \ mW
\end{equation}

Luego, para los diodos $D_{19}$ y $D_{20}$, conjuntamente se calcula una potencia de
\begin{equation} 
	P_{D1} + P_{D2} \approx  \left( 1.7 \ V \right) \frac{30 \ V - 1.7 \ V}{R_{31}} \approx 7.02 \ mW
\end{equation}


\subsection{Etapa de ganancia}
Para el cálculo de la potencia de este bastará con realizar el cálculo\footnote{Los valores de corriente $I_c$ y tensión $V_{ce}$ son los calculados en la etapa de ganancia}:
\begin{align}
P_{ec4}\approx\left[V_{ce-cc} + \frac{\hat{V_{ce-ac}}}{\sqrt{2}}\right]\cdot I_c\approx 540 \ mW
\end{align}
Dado a la potencia media que disipa y la tensión que debe manejar se eligió un TIP41C, el cual puede manejar este trabajo sin disipador.
En cuanto a las otras tensiones $V_{ce}$ de los transistores debido a que no trabajan con señales tan grande se aproxima la potencia disipada a la de continua.
\begin{align}
P_{ec-123}\approx V_{ce-cc} \cdot I_c \approx 597 \ mW
\end{align}
Este transistor todavía se encuentra dentro de la potencia que puede disipar sin disipador.
Por otro lado, para las resistencias de colector de los emisores comunes será una potencia de:
\begin{align}
P_{Rc}= I_c^2 \cdot R_c \approx 0.55 \ W
\end{align}

Para la carga a activa bastará con realizar el producto de la tension de juntura por la corriente.
\begin{align}
P_{CS}\approx\left[V_{ce-cc} + \frac{\hat{V_{ce-ac}}}{\sqrt{2}}\right]\cdot I_c\approx  632\ mW
\end{align}
En cuanto a este transistor cabe destacar que tampoco es necesario el uso de un disipador.
\subsection{Etapa de salida}

Comenzando por la fuente de corriente compensada, se tiene que la potencia que disipa la resistencia que polariza a los diodos es de

\begin{equation}
3k3 \ \Omega \cdot 20 \ mA = 1.32 \ W
\end{equation}

Por lo que se deberá utilizar una resistencia de $2 \ W$. Tanto los diodos como la resistencia que fija la corriente del transistor disiparan muy poca potencia, debido a las pequeñas caídas de tensión sobre estos componentes. El transistor deberá disipar una potencia de

\begin{equation}
I_{c} \cdot (V_{ce_{cc}} + \frac{V_{ce_{ac}}}{\sqrt{2}}) = 7 \ mA \cdot (60V + 37.5V) = 682mW avg
\end{equation}

Continuando con el generador de bias, si bien la corriente que lo atraviesa es de unos $6 \ mA$, el generador de bias como máximo impone una caída de tensión de $4.6 \ V$, por lo que la potencia disipada será poca.

Luego, por la resistencia conectada entre el generador de bias y el rail negativo de alimentación, que vale de $25 \ k\Omega$, circulan alrededor de $2mA$ por lo que se tiene que

\begin{equation}
I^2 \cdot R = 100mW
\end{equation}

Por lo que el resistor puede ser de $250mW$.

Finalmente, pasando a la clase AB, para la cual se decidió utilizar 6 pares de transistores de salida en paralelo luego de analizar las cuentas que se detallarán a continuación pero para $N$ transistores, se tiene que la potencia media para los transistores de salida es

\begin{equation}
(60 \ V - \frac{53 \ V}{\sqrt{2}} - R_e \cdot \frac{13.5 \ A}{6\cdot \sqrt{2}})\cdot \frac{13.5 \ A}{6\cdot \sqrt{2}} = 17.8W avg
\end{equation}

para los transistores driver es

\begin{equation}
(60 \ V - \frac{53 \ V}{\sqrt{2}} - R_e \cdot \frac{13.5 \ A}{ \sqrt{2} \cdot \beta} - 0.7 \ V)\cdot \frac{13.5 \ A}{ \sqrt{2} \cdot \beta} = 1.49 \ W avg
\end{equation}

y para los transistores pre-driver es

\begin{equation}
(60 \ V - \frac{53 \ V}{\sqrt{2}} - R_e \cdot \frac{13.5 \ A}{ \sqrt{2} \cdot \beta^2} - 2 \cdot 0.7 \ V)\cdot \frac{13.5 \ A}{ \sqrt{2} \cdot \beta^2} = 0.144 \ W avg
\end{equation}

Los transistores a utilizar se decidarán luego de contrastar estos valores con los de la simulación, dado que estos transistores serán los componentes que más potencia disiparan en el circuito y se debe de prestar un especial cuidado.

%\end{document}
