%\documentclass[a4paper]{article}
\usepackage[utf8]{inputenc}
\usepackage[spanish, es-tabla, es-noshorthands]{babel}
\usepackage[table,xcdraw]{xcolor}
\usepackage[a4paper, footnotesep = 1cm, width=20cm, top=2.5cm, height=25cm, textwidth=18cm, textheight=25cm]{geometry}
%\geometry{showframe}

\usepackage{tikz}
\usepackage{amsmath}
\usepackage{amsfonts}
\usepackage{amssymb}
\usepackage{float}
\usepackage{graphicx}
\usepackage{caption}
\usepackage{subcaption}
\usepackage{multicol}
\usepackage{multirow}
\setlength{\doublerulesep}{\arrayrulewidth}
\usepackage{booktabs}

\usepackage{hyperref}
\hypersetup{
    colorlinks=true,
    linkcolor=blue,
    filecolor=magenta,      
    urlcolor=blue,
    citecolor=blue,    
}

\newcommand{\quotes}[1]{``#1''}
\usepackage{array}
\newcolumntype{C}[1]{>{\centering\let\newline\\\arraybackslash\hspace{0pt}}m{#1}}
\usepackage[american]{circuitikz}
\usetikzlibrary{calc}
\usepackage{fancyhdr}
\usepackage{units} 

\graphicspath{{../Calculos-Potencia/}{../Caracteristicas/}{../Consideraciones/}{../Gain-Stage/}{../Input-Stage/}{../Output-Stage/}{../Simulaciones/}{../Alimentacion/}{../Conclusiones/}}

\pagestyle{fancy}
\fancyhf{}
\lhead{22.12 Electrónica II}
\rhead{Mechoulam, Lambertucci, Rodriguez, Londero, Scala}
\rfoot{Página \thepage}
%
%\begin{document}
\subsection{Introducción}

\subsection{Simulación de  rendimiento}
El rendimiento esta definido como:
\begin{align}
\eta = \frac{P_{RL}}{P_{vcc}+P_{vee}}
\end{align}
Teniendo en cuenta que la potencia para las señales senoidales se toma la potencia eficaz, siendo esta:
\begin{align}
P_{R}=V_{R-RMS}\cdot I_{R-RMS} =\frac{\hat{V_{R}}}{\sqrt{2}} \cdot \frac{\hat{I_{R}}}{\sqrt{2}}
\end{align}
Finalmente fue simulado el rendimiento obteniendo los siguientes valores:
\begin{align}
\eta=\frac{1083 / 2 W}{ 2\cdot 18 W + 2 \cdot 690 /2}\approx 74.5 \%
\end{align}
\subsection{Cálculo de potencias etapa de ganancia}
En la etapa de ganancia el transistor que mas disipa potencia en cuanto a la señal en modo incremental es el último transistor de los emisores comunes.\footnote{Los valores de corriente $I_c$ y tensión $V_{ce}$ son los calculados en la etapa de ganancia}
Para el cálculo de la potencia de este bastará con realizar el cálculo:
\begin{align}
P_{ec4}\approx\left[V_{ce-cc} + \frac{\hat{V_{ce-ac}}}{\sqrt{2}}\right]\cdot I_c\approx 880mW
\end{align}
En cuanto a las otras tensiones $V_{ce}$ de los transisotres debido a que no trabajan con señales tan grande se aproxima la potencia disipada a la de continua.
\begin{align}
P_{ec-123}\approx V_{ce-cc} \cdot I_c \approx 620mW
\end{align}
En cuanto a las resistencias de colector de los emisores compunes será una potencia de:
\begin{align}
P_{Rc}= I_c^2 \cdot R_c= 0.55 W
\end{align}
 Para la carga a activa bastará con realizar el producto de la tension de juntura por la corriente.
\begin{align}
P_{CS}\approx\left[V_{ce-cc} + \frac{\hat{V_{ce-ac}}}{\sqrt{2}}\right]\cdot I_c\approx 1W
\end{align}
%\begin{align}
%
%\end{align}
%\end{document}
