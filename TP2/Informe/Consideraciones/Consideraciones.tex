
Se debe diseñar un amplificador clase AB de audiofrecuencia para una carga nominal de $8 \ \Omega$. El amplificador debe disipar la máxima potencia a la salida sin recorte para una señal de entrada de $1 V_{pp}$. Se quiere una impedancia de entrada de $50 \ k\Omega$ y corriente de reposo de los transistores de salida ajustable. Además, se consideró inicialmente una potencia máxima de $12 \ W$. Sin embargo, tras consultar acerca de la potencia máxima, fue recomendado disipar una potencia mucho mayor. Se considera finalmente $1 \ kW$ de potencia máxima a la salida sin recorte. El diseño se centra en lograr esta potencia máxima con una THD menor a $1\%$. Además, se debe diseñar para maximizar el rendimiento.
