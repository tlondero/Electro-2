\documentclass[border={1.25cm 0.5cm 0.5cm 0.5cm}, 11pt, tikz, multi=page]{standalone}
\usepackage[utf8]{inputenc}
\usepackage[spanish, es-tabla, es-noshorthands]{babel}

\usepackage[a4paper, footnotesep = 1cm, width=18cm, left=2cm, top=2.5cm, height=25cm, textwidth=18cm, textheight=25cm]{geometry}
%\geometry{showframe}

\usepackage{tikz}
\usepackage{textcomp}
\usetikzlibrary{shapes,arrows}

\usepackage{amsmath}
\usepackage{amsfonts}
\usepackage{amssymb}
\usepackage{float}
\usepackage{graphicx}
\usepackage{caption}
\usepackage{subcaption}
\usepackage{multicol}
\usepackage{multirow}
\setlength{\doublerulesep}{\arrayrulewidth}
\usepackage{booktabs}

\usepackage{array}
\newcolumntype{C}[1]{>{\centering\let\newline\\\arraybackslash\hspace{0pt}}m{#1}}
\usepackage[american]{circuitikz}
\usepackage{fancyhdr}
\usepackage{units}

\begin{document}

%Circuito a color
\begin{page}
\begin{circuitikz}[european voltages]

	%Amplificador error
	\node[color = blue] [op amp, yscale=-1](A){};
	\draw[color = blue]
		(A.+) -- ++(-1,0) node[ground](g1){}
		(A.down) -- ++ (0,0.5) node[ocirc, label=north:$V^{-}$](){}    
		(A.up) -- ++ (0,-0.5) node[ocirc, label=south:$V^{+}$](){}
		(A.-) -- ++ (0,-2.55) node[ocirc](aux1){}
		(A.out) -- ++ (1,0) node[ocirc](aux2){}
		;
		
	%Circuito de protección
	\draw[color = orange]
		(aux2.south) |- ++ (2,-2) node[npn, anchor=C, rotate = 90, xscale=-1, label=below:$T_P$](Q3){}
		(Q3.B) -- ++(0,1.16) to[R, l_=$R_C$] ++ (-2,0) node[ground](){}
		(Q3.E) -- ++(1.5,0) node[](aux4){} to[R, l_=$R_A$] ++ (0,2) node[](aux3){} to[R, l_=$R_B$] ++ (-2,0) -| (Q3.B)
		(aux4.center) -- ++ (0,-1) node[ocirc](aux6){}		
		;
		
	%Transistor de paso
	\draw[color = olive]
		(aux2.north) -- ++ (0,4) node[npn, anchor = B, label=right:$Q_1$](Q1){}
		(Q1.E) -- ++ (2.15,0) node[npn, anchor = B, label=north:$Q_2$](Q2){}
		(Q2.B) -- ++ (1.2,0) node[npn, anchor = B, label=north:$Q_3$](Q3){}
		
		($ (Q1.E) !.5! (Q2.B) $) to[R, l=$R_5$] ++(0,-2) |- ($ (Q3.E) !.5! (aux3) $)% -| (aux3.center)
		(Q2.E) -| (aux3.center)
		(Q1.C) -- ++ (0,1) node[ocirc](aux5){}
		(Q2.C) |- (aux5.east)
		(aux5) ++ (2,0) -| (Q3.C)
		;
				
	%Circuito de detección (Realimentador)
	\draw[color = purple]
		(aux6.west) to[R, l_=$R_1$] (aux1.east)
		;
		
	%Elemento de referencia
	\draw[color = red]
		(aux1.west) to[R, l_=$R_2$] ++ (-2,0) node[](aux7){} to[R, l_=$R_Z$] ++ (-2,0) node[ocirc](aux8){}
		(aux7.center) to[zzDo] ++ (0,-1.5) node[ground](){}
		;
	
	%Fuentes
	\draw
		(aux8.north) to[open] ++ (0,4.5) node[](aux9){} -- ++ (1,0) node[ground](){}
		(aux9.center) to[battery2, l_=$V^{-}$] (aux8.north)
		(aux9.center) ++ (0,3) to[battery2, l_=$V^{+}$] ++ (0,-3) ++ (0,3) |- (aux5.west)

		;	
	
	
	%Carga
	\draw[color = brown]
		(aux6.south) to[R, l=$R_L$] ++ (0,-2) node[ground](){}
		;		
\end{circuitikz}
\end{page}

%FOLDBACK
\begin{page}
\begin{circuitikz}[european voltages]
	\node [circ, label=$V_A$](aux1){};
	\draw
		to[R, l=$R_A$] ++ (2,0) node[label=$V_O$](aux3){} -- ++(0,-1.225) node[npn, anchor=E, yscale=-1, label=right:$T_P$](tp){}
		(aux1) -- ++ (-1,0) node[npn, anchor=E, rotate=90, label=$T_1$](t1){}
		(aux1) to[R, l_=$R_B$] ++ (0,-2) node[circ, label=left:$V_B$](aux2){} to[R, l_=$R_C$] ++ (0,-1.5
) node[ground](){}
		(tp.B) -- (aux2)
		(t1.B) -- ++ (0,-4) -| (tp.C)
		(aux2) ++ (1,1) node[](malla){}
		(aux3.center) -- ++ (0.5,0) node[](muchacaca){}
		;
	\draw[dashed]
		(t1.C) -- ++ (-1,0)
		(muchacaca.center) -- ++ (1,0)
		;
	\draw[->,shift={(malla)}] (210:.7cm) arc (210:-30:.7cm) node at(0,0){};
\end{circuitikz}
\end{page}

%LINEAL
\begin{page}
\begin{circuitikz}[european voltages]
	\node [](aux1){};
	\draw
		to[R, l=$R_A$] ++ (2,0) node[](aux2){} -- ++(0,-0.5) node[npn, anchor=E, yscale=-1, label=right:$T_P$](tp){}
		(tp.B) -| (aux1.center) node[npn, anchor=E, rotate=90, label=$T_1$](t1){}
		(t1.B) |- (tp.C)
		(aux2.center) -- ++ (0.5,0) node[](aux3){}		
		;
	\draw[dashed]
		(t1.C) -- ++ (-1,0)
		(aux3.center) -- ++ (1,0)
		;
\end{circuitikz}
\end{page}

%Circuito BN
\begin{page}
\begin{circuitikz}[european voltages]

	%Amplificador error
	\node [op amp, yscale=-1](A){};
	\draw
		(A.+) -- ++(-1,0) node[ground](g1){}
		(A.down) -- ++ (0,0.5) node[ocirc, label=north:$V^{-}$](){}    
		(A.up) -- ++ (0,-0.5) node[ocirc, label=south:$V^{+}$](){}
		(A.-) -- ++ (0,-2.55) node[ocirc](aux1){}
		(A.out) -- ++ (1,0) node[ocirc](aux2){}
		;
		
	%Circuito de protección
	\draw
		(aux2.south) |- ++ (2,-2) node[npn, anchor=C, rotate = 90, xscale=-1, label=below:$T_P$](Q3){}
		(Q3.B) -- ++(0,1.16) to[R, l_=$R_C$] ++ (-2,0) node[ground](){}
		(Q3.E) -- ++(1.5,0) node[](aux4){} to[R, l_=$R_A$] ++ (0,2) node[](aux3){} to[R, l_=$R_B$] ++ (-2,0) -| (Q3.B)
		(aux4.center) -- ++ (0,-1) node[ocirc](aux6){}		
		;
		
	%Transistor de paso
	\draw
		(aux2.north) -- ++ (0,4) node[npn, anchor = B, label=right:$Q_1$](Q1){}
		(Q1.E) -- ++ (2.15,0) node[npn, anchor = B, label=north:$Q_2$](Q2){}
		(Q2.B) -- ++ (1.2,0) node[npn, anchor = B, label=north:$Q_3$](Q3){}
		
		($ (Q1.E) !.5! (Q2.B) $) to[R, l=$R_5$] ++(0,-2) |- ($ (Q3.E) !.5! (aux3) $)% -| (aux3.center)
		(Q2.E) -| (aux3.center)
		(Q1.C) -- ++ (0,1) node[ocirc](aux5){}
		(Q2.C) |- (aux5.east)
		(aux5) ++ (2,0) -| (Q3.C)
		;
				
	%Circuito de detección (Realimentador)
	\draw
		(aux6.west) to[R, l_=$R_1$] (aux1.east)
		;
		
	%Elemento de referencia
	\draw
		(aux1.west) to[R, l_=$R_2$] ++ (-2,0) node[](aux7){} to[R, l_=$R_Z$] ++ (-2,0) node[ocirc](aux8){}
		(aux7.center) to[zzDo] ++ (0,-1.5) node[ground](){}
		;
	
	%Fuentes
	\draw
		(aux8.north) to[open] ++ (0,4.5) node[](aux9){} -- ++ (1,0) node[ground](){}
		(aux9.center) to[battery2, l_=$V^{-}$] (aux8.north)
		(aux9.center) ++ (0,3) to[battery2, l_=$V^{+}$] ++ (0,-3) ++ (0,3) |- (aux5.west)

		;	
	
	
	%Carga
	\draw
		(aux6.south) to[R, l=$R_L$] ++ (0,-2) node[ground](){}
		;
		
	\draw
		(Q1.B) ++ (0.4,-0.1) node[](cacona1){}
		(Q1.E) ++ (0,0.45) node[](cacona2){}
		(Q3.B) ++ (0.4,-0.1) node[](cacona3){}
		(Q3.E) ++ (0,0.45) node[](cacona4){}
		;
	\begin{scope}[on background layer, very thick,decoration = {markings}, mark = at position 0.05 with {\arrow{>}}] 
    	\draw[line width = 3pt, green!40, postaction = {decorate}, ->]
			(A.out) -| (Q1.B) -| (cacona1.center) -- (cacona2.center) -- (Q1.E) -| (cacona3.center) -- (cacona4.center) -| (aux6.center) -| (A.-) 	
    	;
    \end{scope}
    	
    \begin{scope}[>=latex,color=orange,thick,text=black]
		\draw[rounded corners=7pt]
		(aux1.center) ++ (-0.25,-0.25) arc(-90:90:0.1 and 0.25)
		(aux1.center) ++ (0.25,-0.25) arc(-90:90:-0.1 and 0.25)
		(aux1.center) ++ (-0.25,0.4) arc(180:360:0.25 and 0.1)
		
		(aux6.center) ++ (-0.25,-0.25) arc(-90:90:0.1 and 0.25)
		(aux6.center) ++ (0.25,-0.25) arc(-90:90:-0.1 and 0.25)
		(aux6.center) ++ (-0.25,0.4) arc(180:360:0.25 and 0.1)
		;
    \end{scope}
\end{circuitikz}
\end{page}

%Darlington
\begin{page}
\begin{circuitikz}[european voltages]
	\draw
		(aux2.north) ++ (0,2) -- ++ (0,2) node[npn, anchor = B, label=right:$Q_1$](Q1){}
		(Q1.E) -- ++ (2.15,0) node[npn, anchor = B, label=north:$Q_2$](Q2){}
		(Q2.B) -- ++ (1.2,0) node[npn, anchor = B, label=north:$Q_3$](Q3){}
		
		($ (Q1.E) !.5! (Q2.B) $) to[R, l=$R_5$] ++(0,-2) |- ($ (Q3.E) !.5! (aux3) $) node[](algo){} -- (Q3.E) 
		(Q2.E) -- (Q3.E)
		(Q1.C) -- ++ (0,1) node[](aux5){}
		(Q2.C) |- (aux5.center)
		(aux5) ++ (2,0) -| (Q3.C)
		;
	\draw[dashed]
		(aux2.north) -- ++ (0,2)
		(algo) -- ++ (0,-1.5)
		;

\end{circuitikz}
\end{page}

\begin{page}
\begin{circuitikz}[european voltages]
%Elemento de referencia
	\draw
		node[label=$V_1$](begin){} to[R, l_=$R_2$] ++ (-2,0) node[circ, label=$V_Z$](aux7){} to[R, l_=$R_Z$] ++ (-2,0) node[label=$V^{-}$](aux8){}
		(aux7.center) to[zzDo] ++ (0,-1.5) node[ground](){}
		;
	\draw[dashed]
		(begin) -- ++ (1,0)
		(aux8) -- ++ (-1,0)
		;
\end{circuitikz}
\end{page}

\begin{page}
\begin{circuitikz}[european voltages]
%Elemento de deteccion
	\draw
		node[circ, label=south:$V_0$](begin){} to[R, l_=$R_1$, i_=$i_1$] ++ (-3,0) node[circ, label = south:$0V$](aux7){} 
		;
	\draw[dashed]
		(begin) -- ++ (0,1)
		(aux7) -- ++ (-1,0)
		(aux7) -- ++ (0,1)
		;
\end{circuitikz}
\end{page}

\begin{page}
\begin{circuitikz}[european voltages]
%mierdas
	%Amplificador error
	\node[] [op amp, yscale=-1](A){};
	\draw
		(A.+) -- ++(-1,0) node[ground](g1){}
		(A.down) -- ++ (0,0.5) node[ocirc, label=north:$V^{-}$](){}    
		(A.up) -- ++ (0,-0.5) node[ocirc, label=south:$V^{+}$](){}
		(A.-) to[open] ++ (0,-2.55) node[ocirc](aux1){}
		(aux1) to[short, i=$i_e$] (A.-)
		(A.out) -- ++ (1,0) node[](aux2){}
		;
		
	%Circuito de detección (Realimentador)
	\draw
		(aux6.west) to[R, l_=$R_1$, -*, i=$i_1$] (aux1.center) node[label=south:$0V$]{}
		;
		
	%Elemento de referencia
	\draw
		(aux1.west) to[R, l_=$R_2$, i_=$i_2$] ++ (-2,0) node[](aux7){} to[R, l_=$R_Z$] ++ (-2,0) node[ocirc](aux8){}
		(aux7.center) to[zzDo] ++ (0,-1.5) node[ground](){}
		;
		%Transistor de paso
	\draw
		(aux2.center) -- ++ (0,2) node[npn, anchor = B, label=right:$Q_1$](Q1){}
		(Q1.E) -- ++ (2.15,0) node[npn, anchor = B, label=north:$Q_2$](Q2){}
		(Q2.B) -- ++ (1.2,0) node[npn, anchor = B, label=north:$Q_3$](Q3){}
		
		($ (Q1.E) !.5! (Q2.B) $) to[R, l=$R_5$] ++(0,-2) to[short, -*] ++ (3.11,0)
		(Q2.E) -| (aux3.center)
		(Q1.C) -- ++ (0,1) node[](aux5){}
		(Q2.C) to[short, -*] ++ (0, 1.755)
		(aux5) ++ (-1,0) -| (Q3.C)
		(aux5) ++ (-1,0)  node[ocirc]{}
		(aux3.center) to[short] ++ (0,-3) to[short] ++ (-0.04,0) node[label=south:$V_0$]{}
		;
\end{circuitikz}
\end{page}

%Nuevo ampli
\begin{page}
\begin{circuitikz}
	\node [ground](){};	
	\draw
		to[I, label=$I_N$] ++ (0,2)	to[R, l=$R_{TH}$] ++ (2,0) node[](aux1){} to[R, l_=$R_1$] ++ (0,-2) node[ground](){}
		(aux1.center) -- ++ (2,0) to[R, v^<=$V_1$, l_=$R_i$] ++ (0,-2) node[ground](){}
		(aux1.center) ++ (6,0) node[](aux2){} to[V, v_=$A_{VOL} V_1$] ++ (0,-2) node[ground](){} ++ (0,2) to[R, l=$r_O$] ++ (2,0) to[R, l_=$R_L$, v^=$V_O'$] ++ (0,-2) node[ground](){}
	;		
\end{circuitikz}
\end{page}

\end{document}