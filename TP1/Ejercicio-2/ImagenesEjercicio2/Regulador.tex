\documentclass[border={1.25cm 0.5cm 0.5cm 0.5cm}, 11pt, tikz, multi=page]{standalone}
\usepackage[utf8]{inputenc}
\usepackage[spanish, es-tabla, es-noshorthands]{babel}

\usepackage[a4paper, footnotesep = 1cm, width=18cm, left=2cm, top=2.5cm, height=25cm, textwidth=18cm, textheight=25cm]{geometry}
%\geometry{showframe}

\usepackage{tikz}
\usepackage{textcomp}
\usetikzlibrary{shapes,arrows}

\usepackage{amsmath}
\usepackage{amsfonts}
\usepackage{amssymb}
\usepackage{float}
\usepackage{graphicx}
\usepackage{caption}
\usepackage{subcaption}
\usepackage{multicol}
\usepackage{multirow}
\setlength{\doublerulesep}{\arrayrulewidth}
\usepackage{booktabs}

\usepackage{hyperref}
\hypersetup{
    colorlinks=true,
    linkcolor=blue,
    filecolor=magenta,      
    urlcolor=blue,
    citecolor=blue,    
}

\newcommand{\quotes}[1]{``#1''}
\usepackage{array}
\newcolumntype{C}[1]{>{\centering\let\newline\\\arraybackslash\hspace{0pt}}m{#1}}
\usepackage[american]{circuitikz}
\usepackage{fancyhdr}
\usepackage{units}

%% Definition of blocks:
%\tikzset{%
%  block/.style    = {draw, thick, rectangle, minimum height = 3em,
%    minimum width = 3em},
%  sum/.style      = {draw, circle, node distance = 2cm}, % Adder
%  input/.style    = {coordinate}, % Input
%  output/.style   = {coordinate}, % Output
%  >=Stealth
%}
%
%% Defining string as labels of certain blocks.
%\newcommand{\suma}{\Large $\Sigma$}
%\newcommand{\inte}{$\displaystyle \int$}
%\newcommand{\derv}{\huge $\frac{d}{dt}$}

\begin{document}

%Circuito
\begin{page}
\begin{circuitikz}[european voltages]

	%Amplificador error
	\node[color = blue] [op amp, rotate=90, yscale=-1](A){};
	\draw[color = blue]
		(A.+) node[ground](g1){}
		(A.down) -- ++ (-0.5,0) node[ocirc, label=north:$V^{-}$](){}    
		(A.up) -- ++ (0.5,0) node[ocirc, label=north:$V^{+}$](){}
		(A.-) -- ++ (0,-2) node[ocirc](aux1){}
		(A.out) -- ++ (0,1) node[ocirc](aux2){}
		;
	
	%Pre-regulador
	\draw[color = orange]
		(aux2.west) -- ++(-2,0) node[pnp, anchor = C, rotate=90, label=$Q_3$](Q3){}
		(Q3.E) to[R, l_=$R_7$] ++ (-3,0) -- ++ (0,-1.5) node[ocirc](aux3){} to[D] ++(1.5,0) to[D] ++ (1.5,0) -| (Q3.B) to[R, l=$R_6$] ++ (0,-3) node[ground](){}
		;
		
	%Fuentes
	\draw
		(aux3.south) to[battery2, l_=$V^{+}$] ++ (0,-2.5) node[](aux4){} to[battery2, l_=$V^{-}$] ++ (0,-2.5) node[](aux5){}
		(aux4.center) -- ++ (1,0) node[ground](){}
		;
		
	%Elemento de referencia
	\draw[color = red]
		(aux5) node[ocirc](){}
		(aux5) to[R, l=$R_{10}$] ++ (4,0) node[](aux6){} to[R, l=$R_{9}$] ++ (2,0) -| (aux1.south)
		(aux6.center) to[zzDo] ++ (0,-1.5) node[ground](){}
		;
		
	%Transistor de paso
	\draw[color = olive]
		(aux2.east) -| ++ (5,0) node[npn, anchor = C, label=right:$Q_4$](Q4){}
		(Q4.C) -| (Q4.B)
		(Q4.E) node[npn, anchor = B, label=right:$Q_5$](Q5){}
		(Q5.C) |- (Q4.C)
		(Q5.B) -- (Q4.E)		
		(Q5.E) -- ++ (0,-1) node[ocirc](aux7){}
		(Q4.E) to[R, l=$R_4$] ++(0,-1.5) |- (aux7.west)		
		(Q4.B) -- ++ (-1,0) node[ocirc](aux8){}
		(aux7.west) ++ (-0.75,0) -- ++ (-0.75,0) node[ocirc](aux10){}		
		;
		
	%Circuito de detección (Realimentador)
	\draw[color = purple]
		(aux1.east) to[R, l=$R_8$] ++ (1.75,0) node[ocirc](aux9){}
		;

	%Circuito de protección
	\draw[color = pink]
		(aux8.south) node[npn, anchor = C, label=right:$Q_1$](Q1){}
		(aux9.north) to[R, l=$R_1$] ++(0,3) |- (Q1.B)
		(aux9.south) to[R, l=$R_2$] ++(0,-2) node[ground](){}
		(Q1.E) to[R, l=$R_3$] ++ (0,-3) |- (aux9.east)
		(Q1.E) -| (aux10.north)
		;
		
	%Carga
	\draw[color = brown]
		(aux7.south) to[R, l=$R_L$] ++ (0,-2) node[ground](){}
		;
		
\end{circuitikz}
\end{page}

\end{document}